\documentclass{article}
\usepackage{style} % Required for custom headers

\author{Jorge Gómez Reus}
\date{}
\begin{document}
\maketitle
\begin{enumerate}
	\item ¿Cuáles ingredentes Necesarios para el avance tecnológico?\\
	R: Concepto e Implementación
	\item ¿Cuáles son los autores del trabajo que comúnmente se reconoce como el órigen del campo de las redes neuronales artificales?\\
	R: Warren McCulloch y Walter Pitts
	\item ¿Qué operaciones, en principio, pueden computar las redes neuronales artificiales?\\
	R: Operaciones artiméticas y lógicas
	\item ¿Cuándo se inventó la red de perceptrones y la regla de aprendizaje asociado?\\
	R: En 1950
	\item ¿Qué demostró Frank Rosenblatt con respecto a su red?\\
	R: Su habilidad para ejecutar reconocimiento de protones
	\item ¿Qué usaron Bernard Widrow y Ted Hoff?\\
	R: Un algoritmo para entrendar redes neuronales lineales adaptativas
	\item ¿Por qué la investigación de redes neuronales fue parada?\\
	R: Por las limitaciones de las redes en ese entonces y la falta de computadoras digitales poderosas
	\item ¿Qué desarrollaron Teuvo Kohonen y James Anderson?\\
	R: Redes neuronales que podrían actuar como memorias
	\item ¿A que conceptos se les atribuye la resurección de las redes neuronales artificiales?\\
	R: Uso de mecanismos estadísticos y el algoritmo de 'backprogagation'
	\item ¿Para qué se puede usaba el algoritmo de ``backpropagation''?\\
	R: Para entrenar redes de perceptrones multicapa
	\item ¿Quiénes crearon el algoritmo de ``backpropagation''?\\
	R: David Rumelhart y James McClelland
	\item ¿Quién propuso un mecanismo para aprender de las neuronas biológicas, basado en el condicionamiento de Pavlov?\\
	R: Donald Hebb
	
	
\end{enumerate}

\end{document}
